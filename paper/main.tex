\documentclass[12pt]{article}
\usepackage[utf8]{inputenc}
\usepackage[margin=1in]{geometry}
\usepackage{amsmath,amssymb}
\usepackage{graphicx}
\usepackage{booktabs}
\usepackage{natbib}
\usepackage{setspace}
\usepackage{hyperref}
\usepackage{float}
\usepackage{threeparttable}

\doublespacing

\title{\textbf{Unemployment and Fertility: \\ Evidence from U.S. States}}

\author{Augustin Landier\thanks{HEC Paris. Email: landier@hec.fr. I thank [acknowledgments] for helpful comments.}}

\date{January 2026}

\begin{document}

\maketitle

\begin{abstract}
\noindent This paper estimates the effect of unemployment on birth rates using U.S. state-level panel data from 1995--2023. Exploiting within-state variation in unemployment relative to the national trend, I find that a 1 percentage point increase in unemployment reduces birth rates by 0.21 percent ($t=-4.3$). This estimate is robust to controlling for state GDP per capita and corporate profitability, and the effect persists for five years with gradual attenuation. A placebo test supports a causal interpretation: conditional on current unemployment, future unemployment has no predictive power for current fertility ($t=-1.4$). Cross-sectional variation during the Great Recession corroborates the panel estimates. These findings imply that a typical recession reduces annual births by approximately 36,000, with effects that cumulate over the business cycle.

\vspace{0.3cm}
\noindent \textbf{JEL:} J13, E24, J11 \hspace{1cm} \textbf{Keywords:} Fertility, unemployment, business cycles
\end{abstract}

\newpage

\section{Introduction}

The U.S. total fertility rate has fallen from 2.08 in 1990 to 1.64 in 2023, well below replacement. This decline has important implications for economic growth, fiscal sustainability, and intergenerational transfers. While secular trends---rising female education, delayed marriage, and changing preferences---have received substantial attention, the role of cyclical economic conditions remains less well understood.

This paper examines whether local labor market shocks affect fertility. Using state-level panel data on unemployment and birth rates from 1995--2023, I exploit within-state variation in unemployment to identify the effect of economic conditions on fertility. The identifying assumption is that, conditional on state and year fixed effects, deviations in state unemployment from its trend reflect exogenous labor demand shocks rather than changes in fertility preferences.

The main finding is that a 1 percentage point increase in state unemployment reduces birth rates by 0.21 percent, with a $t$-statistic of $-4.27$. This estimate is economically meaningful: a recession that raises unemployment by 5 percentage points would reduce births by approximately 1 percent, or 36,000 births annually. The effect persists for five years, consistent with unemployment affecting both contemporaneous fertility decisions and planned future births.

I provide evidence supporting a causal interpretation through a placebo test. If the unemployment-fertility relationship reflects causation from unemployment to fertility, then future unemployment (which has not yet occurred) should not predict current fertility after controlling for current unemployment. Consistent with this prediction, the coefficient on future unemployment drops from $-0.0021$ ($t=-4.5$) to $-0.0010$ ($t=-1.4$) when current unemployment is included. This pattern would not obtain if an omitted variable drove both unemployment and fertility.

The Great Recession provides additional variation. States that experienced larger unemployment increases from 2007 to 2010 also experienced larger birth rate declines, and the cross-sectional relationship mirrors the panel estimate.

This paper contributes to the literature on economic conditions and fertility. \citet{currie2014} document that unemployment exposure in early adulthood reduces lifetime fertility, with long-term effects exceeding short-term effects. \citet{schneider2015} examines the Great Recession and finds effects similar in magnitude to mine. I extend this work with a longer sample, controls for state-level economic conditions, and a placebo test that strengthens causal inference.


\section{Data}

\subsection{Sample Construction}

The sample consists of 1,479 state-year observations covering 51 states (including D.C.) from 1995--2023.

\textbf{Birth rates} come from the CDC National Vital Statistics System, which compiles data from all birth certificates filed in the United States. I calculate birth rates as live births per 1,000 women aged 15--44, using population estimates from the Census Bureau.

\textbf{Unemployment rates} come from the Bureau of Labor Statistics' Local Area Unemployment Statistics program, accessed via FRED. I average monthly rates to the annual level.

\textbf{GDP per capita} comes from the Bureau of Economic Analysis, available from 1997. Real GDP (millions of 2017 dollars) is divided by population (thousands) to obtain per capita values.

\textbf{Corporate profitability} is constructed from Compustat North America. For firms headquartered in each state, I calculate the mean return on assets (EBITDA/Assets), requiring at least 10 firms per state-year. This measure proxies for local economic conditions beyond unemployment.

\subsection{Summary Statistics}

Table \ref{tab:summary} presents summary statistics. The mean unemployment rate is 5.25\%, ranging from 1.8\% (North Dakota, 2019) to 13.7\% (Nevada, 2010). Birth rates average 117 per 1,000 women, with substantial cross-state variation (standard deviation of 16.6).

\begin{table}[H]
\centering
\begin{threeparttable}
\caption{Summary Statistics}
\label{tab:summary}
\begin{tabular}{lccccc}
\toprule
 & N & Mean & SD & Min & Max \\
\midrule
Unemployment rate (\%) & 1,479 & 5.25 & 1.93 & 1.80 & 13.68 \\
Birth rate (per 1,000) & 1,479 & 116.8 & 16.6 & 74.4 & 175.7 \\
GDP per capita (\$000s) & 1,377 & 56.2 & 21.9 & 32.1 & 214.2 \\
Mean ROA & 1,314 & 0.039 & 0.053 & $-$0.20 & 0.19 \\
\bottomrule
\end{tabular}
\begin{tablenotes}
\small
\item \textit{Notes:} Sample is state-years from 1995--2023. Birth rate is births per 1,000 women aged 15--44. GDP per capita is in thousands of 2017 dollars. Mean ROA is the average EBITDA/Assets among Compustat firms headquartered in the state (requiring $\geq$10 firms).
\end{tablenotes}
\end{threeparttable}
\end{table}


\section{Empirical Strategy}

I estimate:
\begin{equation}
\log(\text{BirthRate}_{st}) = \beta \cdot \text{Unemployment}_{st} + \mathbf{X}_{st}'\gamma + \mu_s + \delta_t + \varepsilon_{st}
\label{eq:main}
\end{equation}
where $s$ indexes states and $t$ indexes years. The dependent variable is the log birth rate, so $\beta$ represents the semi-elasticity: the percentage change in birth rates per percentage point change in unemployment. $\mathbf{X}_{st}$ includes controls such as log GDP per capita and mean ROA. $\mu_s$ and $\delta_t$ are state and year fixed effects.

The coefficient $\beta$ is identified from within-state, within-year variation. State fixed effects absorb persistent differences in fertility across states (due to demographics, culture, or policy). Year fixed effects absorb national trends (changing norms, federal policy, contraceptive technology). What remains is state-specific deviation from trend---the component I interpret as local labor market shocks.

The identifying assumption is that these residualized unemployment shocks are uncorrelated with unobserved determinants of fertility. This assumption would be violated if, for example, states with declining fertility also experienced declining labor demand for reasons unrelated to fertility. The placebo test in Section \ref{sec:causal} provides evidence against such confounding.

Standard errors are heteroskedasticity-robust. Results are similar with state-clustered standard errors.


\section{Results}

\subsection{Main Estimates}

Table \ref{tab:main} presents the main results. Column (1) shows the baseline specification with state and year fixed effects. A 1 percentage point increase in unemployment is associated with a 0.21\% decline in birth rates ($t=-4.27$). The R-squared of 0.988 indicates that fixed effects absorb nearly all variation; identification comes from residual within-state, within-year variation.

\begin{table}[H]
\centering
\begin{threeparttable}
\caption{Effect of Unemployment on Birth Rates}
\label{tab:main}
\begin{tabular}{lcccc}
\toprule
 & (1) & (2) & (3) & (4) \\
\midrule
Unemployment & $-$0.0021*** & $-$0.0021*** & $-$0.0024*** & $-$0.0024*** \\
 & (0.0005) & (0.0005) & (0.0006) & (0.0006) \\
log(GDP per capita) & & $-$0.006 & $-$0.006 & $-$0.006 \\
 & & (0.008) & (0.013) & (0.013) \\
Mean ROA & & & 0.031** & 0.029* \\
 & & & (0.011) & (0.015) \\
\midrule
$R^2$ & 0.988 & 0.989 & 0.989 & 0.989 \\
Observations & 1,479 & 1,377 & 1,218 & 1,218 \\
State FE & Yes & Yes & Yes & Yes \\
Year FE & Yes & Yes & Yes & Yes \\
\bottomrule
\end{tabular}
\begin{tablenotes}
\small
\item \textit{Notes:} Dependent variable is log(birth rate). Robust standard errors in parentheses. *** $p<0.01$, ** $p<0.05$, * $p<0.1$.
\end{tablenotes}
\end{threeparttable}
\end{table}

Column (2) adds log GDP per capita. The unemployment coefficient is unchanged, and GDP per capita is not significant ($t=-0.70$), suggesting that unemployment captures labor market conditions beyond aggregate output.

Columns (3)--(4) add mean corporate ROA. The unemployment coefficient increases slightly in magnitude to $-$0.0024. Mean ROA enters positively ($t=2.74$), indicating that local economic prosperity---as reflected in firm profitability---independently affects fertility.

\subsection{Magnitude}

A typical recession increases unemployment by 4--5 percentage points. The estimates imply this would reduce birth rates by approximately 1\%. With 3.6 million annual births, this represents roughly 36,000 fewer births per year. Over a five-year recession, cumulative effects could approach 150,000--200,000 births.

\subsection{Dynamic Effects}

Figure \ref{fig:main}, Panel C shows the dynamic response. I estimate separate regressions with unemployment at $t$ predicting birth rates at $t+k$ for $k=0,\ldots,5$. The effect is strongest at $k=0$ through $k=2$ ($\beta \approx -0.0021$), then attenuates to approximately $-0.0015$ by $k=4$--$5$. This persistence suggests unemployment affects both immediate fertility decisions and births planned for the near future.

\subsection{The Great Recession}

Figure \ref{fig:main}, Panel D plots state-level changes in unemployment against changes in birth rates from 2007 to 2010. The cross-sectional relationship is negative, consistent with the panel estimates. Nevada, which experienced the largest unemployment increase (from 4.5\% to 14.9\%), also saw large birth rate declines.


\section{Causal Interpretation}
\label{sec:causal}

A concern is that the unemployment-fertility relationship reflects reverse causation or omitted variables rather than a causal effect of unemployment on fertility. I address this with a placebo test based on future unemployment.

The logic is as follows. If unemployment causes fertility to decline, then current unemployment should predict current fertility, but future unemployment (which has not yet occurred) should have no additional predictive power conditional on current unemployment. In contrast, if an omitted variable drives both unemployment and fertility, future unemployment may predict current fertility even after controlling for current unemployment, because the omitted variable affects both current fertility and future unemployment.

Table \ref{tab:placebo} reports the results. Column (1) shows that future unemployment ($t+1$) predicts current birth rates with $t=-4.48$. However, this reflects the persistence of unemployment: states with high unemployment today tend to have high unemployment next year.

\begin{table}[H]
\centering
\begin{threeparttable}
\caption{Placebo Test: Future Unemployment}
\label{tab:placebo}
\begin{tabular}{lcc}
\toprule
 & (1) & (2) \\
\midrule
Unemployment($t+1$) & $-$0.0021*** & $-$0.0010 \\
 & (0.0005) & (0.0007) \\
Unemployment($t$) & & $-$0.0014* \\
 & & (0.0008) \\
\midrule
$R^2$ & 0.988 & 0.988 \\
Observations & 1,428 & 1,428 \\
\bottomrule
\end{tabular}
\begin{tablenotes}
\small
\item \textit{Notes:} Dependent variable is log(birth rate) at $t$. All specifications include state and year fixed effects. Robust standard errors in parentheses. *** $p<0.01$, ** $p<0.05$, * $p<0.1$.
\end{tablenotes}
\end{threeparttable}
\end{table}

Column (2) is the key specification. After controlling for current unemployment, the coefficient on future unemployment drops to $-0.0010$ with $t=-1.38$ ($p=0.17$). Future unemployment no longer significantly predicts current fertility once current conditions are controlled. This pattern is consistent with a causal interpretation: current unemployment affects fertility, but future unemployment has no independent effect.

\begin{figure}[H]
\centering
\includegraphics[width=\textwidth]{../output/figures/fig_main_results.png}
\caption{Unemployment and Birth Rates}
\label{fig:main}
\begin{minipage}{\textwidth}
\small
\textit{Notes:} Panel A: National time series of birth rates (left axis) and unemployment (right axis), 1995--2023. Panel B: Binned scatter plot of residualized birth rates against residualized unemployment after removing state and year fixed effects; each point represents a vigintile; the fitted line has slope $-0.0021$. Panel C: Coefficients from separate regressions of log birth rates at $t+k$ on unemployment at $t$, for $k=0,\ldots,5$; bars show 95\% confidence intervals. Panel D: Scatter plot of state-level changes in unemployment (horizontal axis) against percentage changes in birth rates (vertical axis) from 2007 to 2010.
\end{minipage}
\end{figure}


\section{Conclusion}

This paper documents that unemployment significantly reduces birth rates. A 1 percentage point increase in state unemployment lowers birth rates by 0.21\%, an effect that persists for five years and is supported by a placebo test showing that future unemployment does not predict current fertility conditional on current unemployment.

These findings have implications for understanding demographic trends and for policy. First, they suggest that business cycle fluctuations have demographic consequences that extend beyond the direct effects on employment and income. Second, the persistence of effects implies that recessions may have long-lasting impacts on cohort sizes, with downstream consequences for education, labor markets, and fiscal systems. Third, policies that stabilize employment during recessions may have benefits for fertility that are not typically considered in cost-benefit analyses.

Several limitations warrant mention. First, I cannot distinguish between postponement and permanent foregone fertility; individual-level data would be needed to track whether delayed births are eventually realized. Second, the Compustat-based profitability measure covers only publicly traded firms and may not fully capture local economic conditions. Third, while the placebo test supports a causal interpretation, it cannot rule out all forms of confounding.

Future research might examine heterogeneity across demographic groups, investigate mechanisms (income effects, uncertainty, marriage market effects), and explore whether the effects vary with the generosity of unemployment insurance or other safety net programs.


\bibliographystyle{apalike}
\bibliography{references}

\end{document}
