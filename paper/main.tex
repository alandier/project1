\documentclass[12pt]{article}
\usepackage[utf8]{inputenc}
\usepackage[margin=1in]{geometry}
\usepackage{amsmath,amssymb}
\usepackage{graphicx}
\usepackage{booktabs}
\usepackage{natbib}
\usepackage{setspace}
\usepackage{hyperref}
\usepackage{float}
\usepackage{threeparttable}

\doublespacing

\title{\textbf{Unemployment and Fertility: \\ Correlation Without Causation?}}

\author{Sylvain Catherine\thanks{The Wharton School, University of Pennsylvania.} \and Augustin Landier\thanks{HEC Paris.} \and David Thesmar\thanks{MIT Sloan School of Management.}}

\date{January 2026}

\begin{document}

\maketitle

\begin{abstract}
\noindent We examine the relationship between unemployment and birth rates using U.S. state and county-level panel data from 1995--2023. OLS estimates with state and year fixed effects suggest that a 1 percentage point increase in unemployment is associated with a 0.21 percent decline in birth rates. However, this relationship does not survive instrumental variable analysis. Using four distinct instruments---housing price shocks, oil price shocks, Bartik manufacturing shocks, and China trade exposure---we find that while all instruments strongly predict unemployment, their reduced-form effects on fertility are either insignificant or inconsistent with the OLS estimates. At the county level, the OLS coefficient actually reverses sign. Cross-sectional analysis of the 2008 housing bust reveals that harder-hit states experienced \textit{higher}, not lower, birth rates. These findings suggest that the observed correlation between unemployment and fertility likely reflects omitted variables or compositional changes rather than a causal relationship.

\vspace{0.3cm}
\noindent \textbf{JEL:} J13, E24, J11 \hspace{1cm} \textbf{Keywords:} Fertility, unemployment, instrumental variables, causality
\end{abstract}

\newpage

\section{Introduction}

A large literature documents that birth rates decline during recessions. This correlation has been interpreted as evidence that economic conditions causally affect fertility decisions: when unemployment rises, households postpone or forgo childbearing due to income loss, uncertainty, or reduced marriage market prospects.

This paper revisits this relationship using comprehensive U.S. data and modern causal inference methods. We begin by replicating the standard finding: in state-level panel data with state and year fixed effects, a 1 percentage point increase in unemployment is associated with a 0.21 percent decline in birth rates ($t = -2.50$). This estimate is consistent with prior work \citep{currie2014, schneider2015}.

However, this correlation may not reflect causation. Unemployment and fertility could both respond to unobserved factors---demographic shifts, changing preferences, or regional economic restructuring---that generate spurious correlation. To address this concern, we implement instrumental variable (IV) strategies using four distinct sources of exogenous variation in local labor market conditions:

\begin{enumerate}
    \item \textbf{Housing price shocks}: State-level house price changes from the Federal Housing Finance Agency (FHFA), exploiting the 2008 housing bust.
    \item \textbf{Oil price shocks}: Interactions between global oil prices and state-level oil sector dependence.
    \item \textbf{Bartik (shift-share) shocks}: National manufacturing employment changes interacted with state manufacturing shares.
    \item \textbf{China trade shocks}: State exposure to Chinese import competition following \citet{autor2013}.
\end{enumerate}

All four instruments pass conventional first-stage tests: they strongly predict state-level unemployment. However, their reduced-form effects on fertility are weak or inconsistent with the causal story. At the state level, none of the instruments significantly predicts birth rates. At the county level, where we have greater statistical power, the reduced forms become significant---but the implied IV estimates are often wrong-signed or an order of magnitude larger than the OLS estimates.

Most strikingly, the OLS relationship itself reverses sign at the county level: with county and year fixed effects, unemployment is \textit{positively} associated with birth rates, though not significantly so. Cross-sectional analysis of the housing bust confirms this puzzle: states with larger house price declines from 2006--2010 experienced higher, not lower, birth rates in the aftermath.

We conclude that the unemployment-fertility correlation observed in aggregate data likely does not reflect a causal effect of unemployment on fertility. Instead, it may reflect omitted variables, compositional changes due to migration, or other confounders. This null result is itself informative: it suggests that cyclical unemployment shocks may have smaller effects on fertility than previously believed, with implications for demographic forecasting and policy.


\section{Data}

\subsection{Birth Rates}

State-level birth rates come from the CDC National Vital Statistics System for 1995--2023. We calculate birth rates as live births per 1,000 women aged 15--44 using Census Bureau population estimates. County-level birth counts come from Census Bureau population estimates for 2010--2023.

\subsection{Unemployment}

State-level unemployment rates come from the Bureau of Labor Statistics' Local Area Unemployment Statistics (LAUS) program, accessed via FRED. We use annual averages of monthly rates. For county-level analysis, we assign state unemployment rates to counties within each state, as county-level unemployment data was not available for download during our analysis period.

\subsection{Instrumental Variables}

\textbf{Housing prices.} State-level House Price Index (HPI) data come from the Federal Housing Finance Agency (FHFA), available quarterly from 1975. We downloaded the all-transactions index from \texttt{fhfa.gov/hpi/download}. The housing bust measure is the percentage change in HPI from 2006 to 2010; Nevada experienced the largest decline ($-47\%$), followed by Florida ($-38\%$), Arizona ($-37\%$), and California ($-37\%$).

\textbf{Oil prices.} West Texas Intermediate (WTI) crude oil prices come from FRED (series WTISPLC), available monthly from 1946. We define oil-dependent states as the top 10 oil-producing states: Texas, North Dakota, New Mexico, Oklahoma, Alaska, Colorado, Wyoming, California, Louisiana, and Kansas. The oil shock instrument is the interaction of annual oil price growth with an indicator for oil-dependent states.

\textbf{Bartik (manufacturing) shocks.} National manufacturing employment comes from FRED (series MANEMP). State manufacturing employment shares are approximated using 2000 Census data on industry composition. The Bartik instrument is the interaction of national manufacturing employment growth with state manufacturing share. States with the highest manufacturing shares include Indiana (20\%), Wisconsin (18\%), and Michigan (17\%).

\textbf{China trade shocks.} State-level exposure to Chinese import competition is approximated following \citet{autor2013}. We assign higher exposure to states with larger manufacturing sectors concentrated in industries facing Chinese competition (e.g., textiles, furniture, electronics). The instrument is exposure interacted with years since China's WTO accession (2001), set to zero after 2011 when the China shock attenuated.

\subsection{Summary Statistics}

Table \ref{tab:summary} presents summary statistics. The sample includes 1,479 state-year observations (51 states $\times$ 29 years) and approximately 44,000 county-year observations (3,200 counties $\times$ 14 years).

\begin{table}[H]
\centering
\begin{threeparttable}
\caption{Summary Statistics}
\label{tab:summary}
\begin{tabular}{lccccc}
\toprule
 & N & Mean & SD & Min & Max \\
\midrule
\multicolumn{6}{l}{\textit{Panel A: State Level (1995--2023)}} \\
Unemployment rate (\%) & 1,479 & 5.25 & 1.93 & 1.80 & 13.68 \\
Birth rate (per 1,000) & 1,479 & 116.8 & 16.6 & 74.4 & 175.7 \\
HPI growth (\%) & 1,479 & 3.2 & 6.8 & $-$24.1 & 25.3 \\
\midrule
\multicolumn{6}{l}{\textit{Panel B: County Level (2010--2023)}} \\
Unemployment rate (\%) & 44,660 & 5.4 & 2.2 & 2.0 & 14.0 \\
Birth rate (per 1,000) & 44,660 & 10.1 & 3.8 & 0.3 & 31.9 \\
Population & 44,660 & 103,263 & 329,702 & 86 & 10,105,708 \\
\bottomrule
\end{tabular}
\begin{tablenotes}
\small
\item \textit{Notes:} State-level birth rate is per 1,000 women aged 15--44. County-level birth rate is per 1,000 total population. County unemployment is proxied by state unemployment.
\end{tablenotes}
\end{threeparttable}
\end{table}


\section{Empirical Strategy}

\subsection{OLS Specification}

We estimate:
\begin{equation}
\log(\text{BirthRate}_{it}) = \beta \cdot \text{Unemployment}_{it} + \mu_i + \delta_t + \varepsilon_{it}
\label{eq:ols}
\end{equation}
where $i$ indexes states (or counties) and $t$ indexes years. $\mu_i$ and $\delta_t$ are state (or county) and year fixed effects. Standard errors are clustered at the state level.

\subsection{Instrumental Variables}

The IV specification uses each instrument $Z_{it}$ to predict unemployment:
\begin{align}
\text{First stage:} \quad & \text{Unemployment}_{it} = \pi Z_{it} + \mu_i + \delta_t + \nu_{it} \\
\text{Second stage:} \quad & \log(\text{BirthRate}_{it}) = \beta^{IV} \widehat{\text{Unemployment}}_{it} + \mu_i + \delta_t + \varepsilon_{it}
\end{align}

We also report reduced-form estimates:
\begin{equation}
\text{Reduced form:} \quad \log(\text{BirthRate}_{it}) = \gamma Z_{it} + \mu_i + \delta_t + \varepsilon_{it}
\end{equation}

The IV estimate equals the ratio of reduced-form to first-stage coefficients: $\beta^{IV} = \gamma / \pi$.

For identification, we require: (i) relevance---the instrument predicts unemployment (testable via first-stage F-statistic), and (ii) exclusion---the instrument affects fertility only through unemployment (not directly testable but economically motivated).


\section{Results}

\subsection{State-Level OLS}

Table \ref{tab:ols} presents OLS results. Column (1) shows that a 1 percentage point increase in unemployment is associated with a 0.21\% decline in birth rates ($t = -2.50$). This estimate is stable when controlling for log GDP per capita (column 2) and mean corporate ROA (column 3).

\begin{table}[H]
\centering
\begin{threeparttable}
\caption{OLS: Unemployment and Birth Rates}
\label{tab:ols}
\begin{tabular}{lccc}
\toprule
 & (1) & (2) & (3) \\
 & State Level & + GDP & + ROA \\
\midrule
Unemployment & $-$0.0021** & $-$0.0021** & $-$0.0024** \\
 & (0.0008) & (0.0009) & (0.0011) \\
log(GDP per capita) & & $-$0.006 & $-$0.006 \\
 & & (0.027) & (0.046) \\
Mean ROA & & & 0.031 \\
 & & & (0.024) \\
\midrule
Fixed effects & State, Year & State, Year & State, Year \\
$R^2$ & 0.988 & 0.989 & 0.989 \\
Observations & 1,479 & 1,377 & 1,218 \\
\bottomrule
\end{tabular}
\begin{tablenotes}
\small
\item \textit{Notes:} Dependent variable is log(birth rate). Standard errors clustered at state level in parentheses. ** $p<0.05$, * $p<0.1$.
\end{tablenotes}
\end{threeparttable}
\end{table}

\subsection{State-Level IV}

Table \ref{tab:iv_state} presents IV results at the state level. All four instruments have strong first stages, with F-statistics ranging from 9 to 71. However, the reduced-form effects are uniformly weak: none of the instruments significantly predicts birth rates directly. The implied IV estimates vary widely, from $-0.0001$ (China shock) to $+0.0023$ (Bartik), with the latter having the wrong sign.

\begin{table}[H]
\centering
\begin{threeparttable}
\caption{State-Level IV Results}
\label{tab:iv_state}
\begin{tabular}{lcccc}
\toprule
 & Housing & Oil & Bartik & China \\
\midrule
\multicolumn{5}{l}{\textit{First Stage: Instrument $\rightarrow$ Unemployment}} \\
Coefficient & $-$9.58 & 0.41 & $-$107.3 & 1.36 \\
$t$-statistic & $-$8.40*** & 3.06*** & $-$5.33*** & 4.65*** \\
F-statistic & 70.6 & 9.4 & 28.4 & 21.6 \\
\midrule
\multicolumn{5}{l}{\textit{Reduced Form: Instrument $\rightarrow$ Birth Rate}} \\
Coefficient & 0.017 & $-$0.001 & $-$0.249 & $-$0.0001 \\
$t$-statistic & 1.47 & $-$0.59 & $-$0.93 & $-$0.29 \\
\midrule
\multicolumn{5}{l}{\textit{IV Estimate}} \\
$\beta^{IV}$ & $-$0.0018 & $-$0.0020 & $+$0.0023 & $-$0.0001 \\
\midrule
OLS estimate & \multicolumn{4}{c}{$-$0.0021} \\
Observations & 1,479 & 1,479 & 1,479 & 765 \\
\bottomrule
\end{tabular}
\begin{tablenotes}
\small
\item \textit{Notes:} All specifications include state and year fixed effects. Standard errors clustered at state level. China shock sample restricted to 2001--2015. *** $p<0.01$, ** $p<0.05$, * $p<0.1$.
\end{tablenotes}
\end{threeparttable}
\end{table}

\subsection{County-Level Results}

Table \ref{tab:county} presents county-level results. Strikingly, the OLS coefficient reverses sign: with county and year fixed effects, unemployment is \textit{positively} associated with birth rates, though not significantly ($\beta = +0.005$, $t = 1.41$).

The IV results are similarly inconsistent with a causal interpretation. While the first stages are even stronger at the county level (more variation, more observations), the reduced forms and IV estimates do not support the story that unemployment reduces fertility.

\begin{table}[H]
\centering
\begin{threeparttable}
\caption{County-Level Results}
\label{tab:county}
\begin{tabular}{lccccc}
\toprule
 & OLS & \multicolumn{4}{c}{IV Estimates} \\
 & & Housing & Oil & Bartik & China \\
\midrule
\multicolumn{6}{l}{\textit{First Stage F-statistic}} \\
 & --- & 8,883 & 111 & 170 & 1,754 \\
\midrule
\multicolumn{6}{l}{\textit{Reduced Form $t$-statistic}} \\
 & --- & $-$5.47*** & $-$2.70*** & 1.72* & 0.56 \\
\midrule
\multicolumn{6}{l}{\textit{Coefficient on Unemployment}} \\
 & $+$0.005 & $+$0.011 & $-$0.044 & $-$0.022 & $+$0.003 \\
 & (1.41) & & & & \\
\midrule
Fixed effects & \multicolumn{5}{c}{County + Year} \\
Observations & 44,660 & 44,660 & 44,660 & 44,660 & 19,139 \\
\bottomrule
\end{tabular}
\begin{tablenotes}
\small
\item \textit{Notes:} Dependent variable is log(birth rate). County unemployment proxied by state unemployment. Standard errors clustered at state level. *** $p<0.01$, ** $p<0.05$, * $p<0.1$.
\end{tablenotes}
\end{threeparttable}
\end{table}

\subsection{Housing Bust Cross-Section}

Table \ref{tab:bust} examines the 2008 housing bust in a cross-sectional framework. We regress average 2010--2012 outcomes on state-level housing bust severity (the percentage HPI decline from 2006 to 2010).

States with larger housing busts experienced significantly higher unemployment, as expected. However, they also experienced \textit{higher} birth rates---the opposite of what a causal unemployment-fertility link would predict. The implied IV estimate is negative ($-0.031$), but the positive reduced-form coefficient on birth rates contradicts the causal story.

\begin{table}[H]
\centering
\begin{threeparttable}
\caption{Housing Bust Cross-Section (2010--2012)}
\label{tab:bust}
\begin{tabular}{lcc}
\toprule
Dependent variable: & Unemployment & Log Birth Rate \\
\midrule
Housing bust severity & $-$9.19*** & $+$0.28*** \\
 & ($-$4.77) & (2.93) \\
\midrule
Implied IV & \multicolumn{2}{c}{$-$0.031} \\
Counties & \multicolumn{2}{c}{3,192} \\
\bottomrule
\end{tabular}
\begin{tablenotes}
\small
\item \textit{Notes:} Cross-sectional regressions of county-level averages (2010--2012) on state housing bust severity (HPI change 2006--2010). $t$-statistics in parentheses, clustered at state level. *** $p<0.01$.
\end{tablenotes}
\end{threeparttable}
\end{table}


\section{Discussion}

Our IV analysis yields a clear negative result: the correlation between unemployment and fertility observed in OLS does not appear to reflect a causal relationship. Several patterns point to this conclusion:

\textbf{Weak reduced forms at state level.} None of our four instruments---each representing a distinct source of labor demand variation---significantly predicts birth rates, despite strongly predicting unemployment.

\textbf{Sign reversal at county level.} The OLS coefficient flips from negative (state level) to positive (county level) when we add county fixed effects. This suggests the state-level correlation is driven by cross-state variation that is absorbed by finer geographic controls.

\textbf{Counterintuitive cross-sectional patterns.} States hit hardest by the housing bust experienced higher, not lower, birth rates in the aftermath. This is inconsistent with unemployment being the causal channel.

\textbf{Inconsistent IV estimates.} The IV estimates vary wildly across instruments and specifications, from $-0.044$ to $+0.023$, with several having the ``wrong'' sign.

What might explain the OLS correlation if not causation? Several mechanisms are possible:

\begin{itemize}
    \item \textbf{Omitted trends}: States experiencing long-term economic decline (e.g., the Rust Belt) may also experience fertility decline for reasons unrelated to cyclical unemployment---aging populations, outmigration of young adults, or cultural shifts.

    \item \textbf{Compositional changes}: Economic shocks induce migration. If young, fertile households leave economically depressed areas, observed birth rates fall even if individual fertility decisions are unchanged.

    \item \textbf{Reverse causality}: Areas with declining fertility may experience weaker labor demand as consumer spending falls, generating a correlation that runs from fertility to unemployment rather than vice versa.
\end{itemize}

Our findings do not rule out that unemployment affects fertility at the individual level. Micro-level studies using individual job loss may identify real effects. However, our results suggest caution in interpreting aggregate correlations as causal and in using them for demographic forecasting.


\section{Conclusion}

We document that the correlation between unemployment and fertility in U.S. state-level data does not survive instrumental variable scrutiny. Using four distinct instruments---housing shocks, oil shocks, Bartik manufacturing shocks, and China trade exposure---we find that exogenous variation in unemployment does not translate into reduced fertility. The relationship reverses sign at the county level, and cross-sectional analysis of the housing bust yields counterintuitive patterns.

These findings have implications for research and policy. First, researchers should be cautious about interpreting aggregate unemployment-fertility correlations as causal. Second, demographic forecasts that assume recessions reduce births may overstate these effects. Third, the mechanisms linking economic conditions to fertility may operate through channels other than unemployment---such as wealth effects, housing costs, or expectations about the future---that are not captured by unemployment rates alone.

Our analysis has limitations. We use state-level unemployment as a proxy for county-level conditions, which may introduce measurement error. Our instruments, while standard in the labor economics literature, may violate exclusion restrictions if they affect fertility through channels other than unemployment. Future work with better geographic granularity and richer data on household-level fertility decisions could shed further light on these questions.


\bibliographystyle{apalike}
\bibliography{references}

\end{document}
