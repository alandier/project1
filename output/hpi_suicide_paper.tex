\documentclass[12pt]{article}
\usepackage[margin=1in]{geometry}
\usepackage{amsmath,amssymb}
\usepackage{booktabs}
\usepackage{graphicx}
\usepackage{float}
\usepackage{hyperref}
\usepackage{natbib}
\usepackage{setspace}
\usepackage{caption}
\usepackage{threeparttable}
\doublespacing

\title{\Large \textbf{Housing Wealth and Suicide}\thanks{We thank [acknowledgments] for helpful comments and suggestions. Catherine: Wharton School, University of Pennsylvania (\texttt{scath@wharton.upenn.edu}). Landier: HEC Paris. Thesmar: MIT Sloan.}}

\author{
Sylvain Catherine \and
Augustin Landier \and
David Thesmar
}

\date{\today}

\begin{document}

\maketitle

\begin{abstract}
\noindent We estimate the causal effect of housing wealth on suicide rates. Using state-level U.S. data from 2000--2022, we instrument local house price growth with land supply elasticity interacted with national housing cycles. A 10 percentage point decline in house prices increases suicide rates by 9 percent. Our estimates imply that the 2007--2011 housing bust caused approximately 10,000--13,000 additional suicide deaths. These findings provide evidence that wealth shocks contribute to ``deaths of despair.''

\vspace{0.3cm}
\noindent \textbf{JEL Codes}: I12, R31, J11

\noindent \textbf{Keywords}: Suicide, Housing Wealth, Deaths of Despair, Instrumental Variables
\end{abstract}

\newpage

\section{Introduction}

Suicide rates in the United States have increased by over 30 percent since 2000, now claiming nearly 50,000 lives annually. This rise has been particularly pronounced among middle-aged Americans, contributing to the broader phenomenon of ``deaths of despair'' \citep{case2015rising, case2020deaths}. While economic distress is often implicated, establishing causality is difficult: economic conditions, mental health, and mortality are jointly determined.

This paper estimates the causal effect of housing wealth shocks on suicide. Housing is the primary asset for most households, and the 2000s housing cycle generated large, geographically concentrated wealth shocks. National house prices rose 80 percent from 2000 to 2006, then fell 30 percent by 2012, with dramatically different experiences across states.

We exploit geographic variation in housing supply elasticity for identification. Following \cite{saiz2010geographic}, we instrument state-level house price growth using land supply constraints interacted with national housing cycles. States with inelastic supply---coastal areas with geographic barriers and restrictive zoning---experience amplified price swings in response to national demand shocks. This isolates variation driven by national factors and time-invariant geography, rather than local conditions that might directly affect suicide.

Our IV estimate implies that a 10 percentage point house price decline increases suicide by 8.9 percent, with a t-statistic of 4.1. The effect is six times larger than OLS, suggesting attenuation from measurement error. The housing bust of 2007--2011 caused approximately 10,000--13,000 excess suicides---about 6--7 percent of the total during this period.

\subsection{Related Literature}

\textbf{Deaths of despair.} \cite{case2015rising} document rising mortality among middle-aged white Americans from suicide, drug overdoses, and alcohol-related liver disease. \cite{case2020deaths} attribute this to economic and social decline in working-class communities. \cite{charles2019manufacturing}, \cite{pierce2020trade}, and \cite{autor2019work} show that manufacturing decline and trade shocks increase mortality. We provide evidence that wealth shocks---distinct from labor market shocks---also matter.

\textbf{Health effects of economic conditions.} \cite{ruhm2000good} finds recessions reduce mortality by decreasing traffic, pollution, and unhealthy behaviors. However, \cite{sullivan2009job} show job displacement increases mortality, particularly from suicide. \cite{currie2015housing} document that foreclosures harm health. We complement this work by isolating the effect of house price changes, which affect wealth even absent foreclosure.

\textbf{Real effects of house prices.} House prices affect consumption \citep{mian2013household}, fertility \citep{dettling2014house, lovenheim2013effect}, entrepreneurship \citep{adelino2015house}, and labor supply \citep{disney2010house}. We extend this literature to mortality, showing that housing wealth has life-or-death consequences.

\section{Data}

\textbf{Suicide rates.} We obtain state-level age-adjusted suicide rates per 100,000 from CDC WONDER, using ICD-10 codes X60--X84 (intentional self-harm). Our sample covers 2000--2022 for 50 states plus DC ($N = 1,173$).

\textbf{House prices.} We use the FHFA all-transactions House Price Index at the state level, computing annual growth rates from Q4 values.

\textbf{Supply elasticity.} We construct state-level elasticity measures from \cite{saiz2010geographic}, who estimates MSA-level elasticities using geographic constraints (water, slopes) and regulation. Low-elasticity states include California (0.6), Hawaii (0.5), and the Northeast corridor. High-elasticity states include Texas (2.0), Kansas (2.3), and the Great Plains.

Table \ref{tab:sumstats} reports summary statistics. The average suicide rate is 14.1 per 100,000, with substantial cross-state variation (SD = 4.4). Annual HPI growth averages 4.6\% with large swings (SD = 6.3\%, range $-24\%$ to $+22\%$).

\begin{table}[H]
\centering
\caption{Summary Statistics}
\label{tab:sumstats}
\begin{tabular}{lcccc}
\toprule
Variable & Mean & SD & Min & Max \\
\midrule
Suicide rate (per 100k) & 14.12 & 4.39 & 5.65 & 32.31 \\
HPI growth (\%) & 4.63 & 6.33 & $-23.78$ & 22.01 \\
Land elasticity & 1.58 & 0.59 & 0.40 & 2.80 \\
Instrument & 3.45 & 3.74 & $-9.75$ & 18.78 \\
\bottomrule
\multicolumn{5}{l}{\footnotesize \textit{Notes}: $N = 1{,}173$ state-years. 51 states, 2000--2022.}
\end{tabular}
\end{table}

Figure \ref{fig:trends} plots national trends. Suicide rates were flat from 2000--2007, then rose steadily---precisely when house prices peaked and began declining.

\begin{figure}[H]
\centering
\includegraphics[width=\textwidth]{figures/fig1_time_series.pdf}
\caption{\textbf{National Trends, 2000--2022.} Both series indexed to 2000 = 100. Shaded area: GFC (2007--2009). House prices peaked in 2006. Suicide rates began rising in 2007.}
\label{fig:trends}
\end{figure}

\section{Empirical Strategy}

We estimate:
\begin{equation}
\log(\text{Suicide}_{st}) = \beta \cdot \text{HPIGrowth}_{st} + \gamma_s + \delta_t + \varepsilon_{st}
\end{equation}
where $\gamma_s$ and $\delta_t$ are state and year fixed effects. Standard errors are clustered by state.

OLS may be biased by omitted local conditions or reverse causality. We instrument HPI growth with:
\begin{equation}
Z_{st} = \Delta \text{HPI}^{\text{National}}_t \times \frac{1}{\text{Elasticity}_s}
\end{equation}

The instrument isolates variation in local prices driven by national demand shocks hitting states differentially based on time-invariant supply constraints. Year fixed effects absorb direct effects of national shocks; state fixed effects absorb level differences across states. Identification comes from the \textit{interaction}: how much each state's prices move relative to national trends, driven by geography.

\textbf{Exclusion restriction.} Our identifying assumption is that land supply elasticity affects suicide only through house prices. This could be violated if elasticity correlates with other time-varying determinants of suicide. We address this concern in three ways. First, elasticity is determined by geography (mountains, water, buildable land) measured decades ago, making it unlikely to correlate with recent changes in suicide risk factors. Second, Figure \ref{fig:pretrends} shows that low- and high-elasticity states had parallel suicide trends before 2007, suggesting no pre-existing differential trends. Third, results are robust to controlling for state unemployment rates, which could confound if elastic-supply states had different labor market trajectories.

\begin{figure}[H]
\centering
\includegraphics[width=0.85\textwidth]{figures/fig_pretrends.pdf}
\caption{\textbf{Pre-trends: Low vs High Elasticity States.} Suicide rates indexed to 2000 = 100. Low-elasticity states (CA, FL, NY, etc.) and high-elasticity states (TX, KS, IA, etc.) had parallel trends before the housing bust. Divergence occurs during 2007--2011, when low-elasticity states experienced larger HPI declines.}
\label{fig:pretrends}
\end{figure}

\section{Results}

\subsection{Main Estimates}

Figure \ref{fig:binscatter} shows a binned scatterplot of residualized suicide against residualized HPI growth. The negative slope is clear: state-years with above-average house price growth have below-average suicide rates.

\begin{figure}[H]
\centering
\includegraphics[width=0.75\textwidth]{figures/fig2_binscatter.pdf}
\caption{\textbf{House Prices and Suicide.} Binned scatterplot after partialling out state and year fixed effects. Each point is a ventile mean. $N = 1,173$.}
\label{fig:binscatter}
\end{figure}

Table \ref{tab:main} reports estimates. OLS yields $\hat{\beta} = -0.0015$ (t = $-2.3$): a 10pp HPI decline is associated with 1.5\% higher suicide. The reduced form (Column 2) shows that the instrument strongly predicts suicide (t = $-4.1$). The first-stage F-statistic is 77. The IV estimate (Column 4) is $-0.0089$: a 10pp decline causes 8.9\% higher suicide.

\begin{table}[H]
\centering
\caption{Effect of House Prices on Suicide}
\label{tab:main}
\begin{threeparttable}
\begin{tabular}{lcccc}
\toprule
 & (1) OLS & (2) Reduced Form & (3) First Stage & (4) IV \\
\midrule
HPI Growth & $-0.00149^{**}$ &  &  & $-0.00889^{***}$ \\
 & (0.00064) &  &  & (0.00217) \\
Instrument &  & $-0.00377^{***}$ & $0.424^{***}$ &  \\
 &  & (0.00092) & (0.048) &  \\
\midrule
State, Year FE & Yes & Yes & Yes & Yes \\
Observations & 1,173 & 1,173 & 1,173 & 1,173 \\
First-stage F &  &  & 77 & 77 \\
\bottomrule
\end{tabular}
\begin{tablenotes}
\small
\item \textit{Notes}: Dependent variable in (1), (2), (4) is log suicide rate. Standard errors clustered by state. *** p$<$0.01, ** p$<$0.05.
\end{tablenotes}
\end{threeparttable}
\end{table}

The IV estimate exceeds OLS by a factor of six. This is consistent with attenuation bias: state-level HPI imperfectly measures the housing wealth shocks experienced by individual residents (who vary in homeownership, equity, and location).

\subsection{Magnitude}

We calculate excess deaths as follows. The average state experienced a 13.5 percentage point cumulative HPI decline from 2006 to 2011 (Nevada fell 55pp; California fell 40pp). Our IV coefficient of $-0.0089$ implies that each percentage point decline increases suicide by 0.89\%. Thus, the average state experienced a $13.5 \times 0.89\% = 12\%$ increase in suicide rates attributable to house price declines.

Applying this proportional effect to the approximately 175,000 suicides that occurred during 2007--2011 yields roughly 11,000 excess deaths. Using state-specific HPI declines and weighting by population gives a similar estimate of 10,000--13,000 excess suicides---approximately 6--7\% of total suicides during this period, or about 2,000--2,500 per year.

\subsection{Robustness}

Table \ref{tab:robust} presents robustness checks. Panel A shows results are stable when excluding potential outliers: DC (extreme values due to size and low elasticity), California (largest state), or high-suicide rural states (Alaska, Wyoming, Montana).

Panel B adds state unemployment rates as a control. If our instrument captured labor market shocks rather than wealth effects, controlling for unemployment should attenuate the estimate. Instead, the IV coefficient strengthens slightly to $-0.0098$ (RF $t = -4.16$), supporting the interpretation that housing wealth operates through channels distinct from employment.

Panel C examines heterogeneity by time period. The effect is strongest during the bust period (2007--2011), when HPI variation was largest, though confidence intervals overlap across periods.

\begin{table}[H]
\centering
\caption{Robustness}
\label{tab:robust}
\begin{tabular}{lccc}
\toprule
Specification & IV Estimate & RF $t$ & N \\
\midrule
\multicolumn{4}{l}{\textit{Panel A: Sample restrictions}} \\
Baseline & $-0.00889$ & $-4.09$ & 1,173 \\
Excluding DC & $-0.00932$ & $-5.70$ & 1,150 \\
Excluding CA & $-0.00961$ & $-3.81$ & 1,150 \\
Excluding AK, WY, MT & $-0.00947$ & $-4.31$ & 1,104 \\
\midrule
\multicolumn{4}{l}{\textit{Panel B: Controls}} \\
+ Unemployment rate & $-0.00978$ & $-4.16$ & 1,122 \\
\midrule
\multicolumn{4}{l}{\textit{Panel C: By time period}} \\
2000--2006 (Boom) & $-0.01126$ & $-2.08$ & 357 \\
2007--2011 (Bust) & $-0.00694$ & $-3.03$ & 255 \\
2012--2022 (Recovery) & $-0.00756$ & $-2.27$ & 561 \\
\bottomrule
\end{tabular}
\end{table}

\subsection{Fertility: A Null Result}

As a specification check, we examine fertility. OLS shows a positive correlation between HPI growth and birth rates (t = 2.2), consistent with \cite{dettling2014house}. However, the IV estimate is zero (t = $-1.0$), indicating no causal effect. The OLS correlation likely reflects selection (families planning children move to appreciating areas). The contrast with suicide---where IV strengthens the finding---supports our identification.

\section{Discussion}

\textbf{Mechanisms.} The effect likely operates through financial distress from home equity loss, foreclosure risk, and underwater mortgages. Housing represents over 60\% of wealth for the median homeowner, making house prices the dominant determinant of household wealth.

\textbf{Policy implications.} Programs stabilizing house prices or assisting underwater homeowners (e.g., HAMP, HARP during the GFC) may have mortality benefits. More broadly, our findings support the view that economic distress causally contributes to deaths of despair, strengthening the case for financial safety nets.

\textbf{Limitations.} State-level analysis may mask heterogeneity. We cannot distinguish homeowner vs. renter effects, or wealth effects from foreclosure effects. Future work with individual-level data could address these questions.

\section{Conclusion}

Housing wealth shocks causally affect suicide rates. A 10 percentage point house price decline increases suicide by 9\%. The 2007--2011 housing bust caused an estimated 10,000--13,000 excess deaths. These findings demonstrate that wealth shocks---not just labor market shocks---contribute to deaths of despair.

\newpage
\singlespacing
\bibliographystyle{aer}
\begin{thebibliography}{99}

\bibitem[Adelino et al.(2015)]{adelino2015house}
Adelino, M., Schoar, A., \& Severino, F. (2015). House prices, collateral, and self-employment. \textit{Journal of Financial Economics}, 117(2), 288--306.

\bibitem[Autor et al.(2019)]{autor2019work}
Autor, D., Dorn, D., \& Hanson, G. (2019). When work disappears: Manufacturing decline and the falling marriage market value of young men. \textit{American Economic Review: Insights}, 1(2), 161--178.

\bibitem[Case \& Deaton(2015)]{case2015rising}
Case, A., \& Deaton, A. (2015). Rising morbidity and mortality in midlife among white non-Hispanic Americans in the 21st century. \textit{Proceedings of the National Academy of Sciences}, 112(49), 15078--15083.

\bibitem[Case \& Deaton(2020)]{case2020deaths}
Case, A., \& Deaton, A. (2020). \textit{Deaths of Despair and the Future of Capitalism}. Princeton University Press.

\bibitem[Charles et al.(2019)]{charles2019manufacturing}
Charles, K. K., Hurst, E., \& Schwartz, M. (2019). The transformation of manufacturing and the decline of US employment. \textit{NBER Macroeconomics Annual}, 33(1), 307--372.

\bibitem[Currie \& Tekin(2015)]{currie2015housing}
Currie, J., \& Tekin, E. (2015). Is there a link between foreclosure and health? \textit{American Economic Journal: Economic Policy}, 7(1), 63--94.

\bibitem[Dettling \& Kearney(2014)]{dettling2014house}
Dettling, L. J., \& Kearney, M. S. (2014). House prices and birth rates: The impact of the real estate market on the decision to have a baby. \textit{Journal of Public Economics}, 110, 82--100.

\bibitem[Disney et al.(2010)]{disney2010house}
Disney, R., Gathergood, J., \& Henley, A. (2010). House price shocks, negative equity, and household consumption in the United Kingdom. \textit{Journal of the European Economic Association}, 8(6), 1179--1207.

\bibitem[Lovenheim \& Mumford(2013)]{lovenheim2013effect}
Lovenheim, M. F., \& Mumford, K. J. (2013). Do family wealth shocks affect fertility choices? Evidence from the housing market. \textit{Review of Economics and Statistics}, 95(2), 464--475.

\bibitem[Mian et al.(2013)]{mian2013household}
Mian, A., Rao, K., \& Sufi, A. (2013). Household balance sheets, consumption, and the economic slump. \textit{Quarterly Journal of Economics}, 128(4), 1687--1726.

\bibitem[Pierce \& Schott(2020)]{pierce2020trade}
Pierce, J. R., \& Schott, P. K. (2020). Trade liberalization and mortality: Evidence from US counties. \textit{American Economic Review: Insights}, 2(1), 47--64.

\bibitem[Ruhm(2000)]{ruhm2000good}
Ruhm, C. J. (2000). Are recessions good for your health? \textit{Quarterly Journal of Economics}, 115(2), 617--650.

\bibitem[Saiz(2010)]{saiz2010geographic}
Saiz, A. (2010). The geographic determinants of housing supply. \textit{Quarterly Journal of Economics}, 125(3), 1253--1296.

\bibitem[Sullivan \& von Wachter(2009)]{sullivan2009job}
Sullivan, D., \& von Wachter, T. (2009). Job displacement and mortality: An analysis using administrative data. \textit{Quarterly Journal of Economics}, 124(3), 1265--1306.

\end{thebibliography}

\end{document}
